\subsection{States, hashes, messages, representatives}

Light node protocol states are also sets of messages,
each message being a triple $(c, v, j)$, where:
\begin{itemize}
    \item $c$ is a (proposed) consensus value;
    \item $v$ identifies the message sender;
    \item $j$, the justification, is the {\em set of hashes} of all messages
        of the protocol state seen by the sender at the time of message
        sending.
\end{itemize}

The fact that justifications are sets of hashes allows for a simpler,
non-recursive definition of states. To begin, the type of hashes can be any
totally ordered set. In turn, a justification is a list of hashes, and
a message is a triple:

\begin{coq}
Variable hash : Type.
Variable (about_H : `{StrictlyComparable hash}).
Definition justification_type : Type := list hash.
Definition message : Type := C * V * justification_type.
\end{coq}

The total order on hashes induces a total lexicographic order on
justifications, which can be extended to messages.

We can therefore work with sorted lists of hashes as representatives
for sets of hashes, reducing equality between justifications to
syntactic equality.

We define state equality set equality, that is,
double inclusion between the sets of messages representing states.

We assume an injective function from messages to hashes:
\begin{coq}
Parameters (hash_message : message -> hash)
           (hash_message_injective : Injective hash_message).
\end{coq}
This allows us to recursively define a function \verb"hash_state" taking
states and returning sorted lists of hashes, i.e., justifications, with the
property that two justifications are equal iff they belong to
states which are equal as sets:

\begin{coq}
Lemma hash_state_injective : forall sigma1 sigma2,
  hash_state sigma1 = hash_state sigma2
  <->
  set_eq sigma1 sigma2.
\end{coq}

This allows for the following inductive definition of protocol states:

\begin{coq}
Inductive protocol_state : state -> Prop :=
| protocol_state_nil : protocol_state state0
| protocol_state_cons : forall (j : state),
    protocol_state j ->
    forall (c : C),
      valid_estimate c j ->
      forall (v : V) (s : state),
        In (c, v, hash_state j) s ->
        protocol_state (set_remove compare_eq_dec
        													(c, v, hash_state j) s) ->
        NoDup s ->
        not_heavy s ->
        protocol_state s.
\end{coq}

The above definition reads as:
\begin{itemize}
    \item a protocol state is either empty; or
    \item it is a duplicate-free state \verb|s| that does not exceed the fault tolerance threshold for which
        there exists a consensus value \verb|c|, a sender \verb|v|, and a state \verb|j| such that:
        \begin{itemize}
            \item \verb|j| is a protocol state;
            \item \verb|c| is a consensus value which the estimator agrees on for \verb|j|;
            \item \verb|(c,v, hash_state j)| belongs to \verb|s|;
            \item The state obtained from \verb|s| by removing
                \verb|(c,v, hash_state j)| is a protocol state.
        \end{itemize}
\end{itemize}